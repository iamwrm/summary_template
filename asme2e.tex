
%%% use twocolumn and 10pt options with the asme2e format
\documentclass[12pt,a4]{article}

\usepackage[letterpaper]{geometry}
\usepackage{times}
\geometry{top=1.0in, bottom=1.0in, left=1.0in, right=1.0in}

\linespread{1.9}% in order to get double space in word 
				% after converting

%%%%%%%%%%%%%
\begin{document}


\begin{flushleft}
Ren, Wang\\
\today\\
VG100\\
Professor Cynthia Vagnetti \\
\end{flushleft}

\begin{center}
Summary
\end{center}


In the article, Butterfield \cite{but} states that engineers support the modern
society to function well by ensuring its fundamental facilities. 
The author demonstrates his thesis through three strands considering
infrastructure or the society, which are ``connect'', ``supply'' and
``protect''. 
By designing and building infrastructure, engineers influence the society on
various levels, from small family unit to the whole globe.  
%
%
% Connect
For the first strand ``Connect'', Butterfield claims that engineers bind
communities together through transport infrastructure and telecommunication
technology.
The author asserts his view on transport infrastructure by giving the evidence
about expressway, railway and aeroplane which guarantee people's ability to
travel wherever they want and reduce the time wasted. 
Engineers also improve the reliably of consumables distribution, contributing
significantly to the society's stable functioning. 
The modern telecommunication technology such as 4G networks make people able to
share their ideas with others faster, resolving the distance gap. 
%
%
% Supply
For the second strand ``Supply'', Butterfield states that engineers develop and
build gas, water electricity and sewage infrastructure.
These systems not only make the society develop and grow, but also sustain
society by ensuring every people's lowest everyday needs and improving
everyone's living standards. 
Engineers' skill of designing, building and maintaining this infrastructure
plays an important role in assuring its consistent reliability.
Then Butterfield emphasizes the importance of this basic infrastructure by
illustrating the fact that this infrastructure can easily become disfunction by
natural disaster or corrupted government.
By giving the example of the inadequate investment on energy infrastructure from
British government in recent years, the author proves the point that no one
should take nowadays adequate supply infrastructure for granted. 
%
% Protect
%
For the third strand ``Protect'', Butterfield claims that engineers protect
society in many ways.
The author proves his thesis by giving the example that engineers build flood
defences to protect people and their houses near rivers, consequently avoiding
further damage to the local communities and private property.
Then the author gives another example about engineers protecting historic
buildings and constructions.
Preserving historic buildings also have benefits to culture and society, for the
new generation needs something to enjoy and learn from.
Engineers also contribute a lot in reducing greenhouse gas emissions.
%
% How do they contribute to social and cultural development?
% What significant example is provided? 
%
The author gives the view that planing and fore-thought are the key to the
success of the project.
Engineers check and eliminate faults by physical or mathematical modelling
before the technology or the infrastructure is mass-produced and opened to the
public.
Engineers also have the view to forsee what is needed for the future.
%
% Society
%
Engineers create chances for an open society.
Thanks to engineers, people are able to cross the barriers of distance and
communicate with others.
Engineers have made it easier to share ideas, meet with friends etc.
People are easier to be participate in society.
Then Butterfield gives an example of the glass dome over the parliamentary
building, which express the 'openness' of modern German politics.
This is a good example of how engineers contribute to social development.
The author argues that modern governments can not be stable unless the basic
requirements of the population are met.
Law and order are maintained because the presence of infrastructure built by
engineers.
%
%
Engineers remain their pride in their works and field, unwilling to be under the
spotlight on purpose.
% 
%
In the end, the author concludes that without engineers, the society cannot
function effectively.
Besides engineers' invaluable contribution to infrastructure, they also
influence society's development direction collectively by resolving physical
gaps, influencing the authority and emphasizing the priority of sustainability.
%
%
%
%
%
% No less than 450 words and no more than 500 words

\newpage
\bibliographystyle{asmems4}
\bibliography{asme2e}


\end{document}
