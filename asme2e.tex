
%%% use twocolumn and 10pt options with the asme2e format
\documentclass[12pt,a4]{article}

\usepackage[letterpaper]{geometry}
\usepackage{times}
\geometry{top=1.0in, bottom=1.0in, left=1.0in, right=1.0in}

\linespread{1.9}% in order to get double space in word 
				% after converting

%%%%%%%%%%%%%
\begin{document}


\begin{flushleft}
Ren, Wang\\
\today\\
VG100\\
Professor Cynthia Vagnetti \\
\end{flushleft}

\begin{center}
Summary
\end{center}


In the article, Butterfield \cite{but} states that engineers support the modern
society to function well by ensuring its fundamental facilities. 
The author demonstrates his thesis through three strands considering
infrastructure or the society, which are ``connect'', ``supply'' and
``protect''. 
By designing and building infrastructure, engineers influence the society on
various levels, from small family unit to the whole globe.  
%
%
% Connect
For the first strand ``Connect'', Butterfield claims that engineers bind
communities together through transport infrastructure and telecommunication
technology.
The author asserts his view on transport infrastructure by giving the evidence
about expressway, railway and aeroplane which guarantee people's ability to
travel wherever they want and reduce the time wasted. 
Engineers also improve the reliably of consumables distribution, contributing
significantly to the society's stable functioning. 
The modern telecommunication technology such as 4G networks make people able to
share their ideas with others faster, resolving the distance gap. 
%
%
% Supply
For the second strand ``Supply'', Butterfield states that engineers develop and
build gas, water electricity and sewage infrastructure.
These systems not only make the society develop and grow, but also sustain
society by ensuring every people's lowest everyday needs and improving
everyone's living standards. 
Engineers' skill of designing, building and maintaining this infrastructure
plays an important role in assuring its consistent reliability.
Then Butterfield emphasizes the importance of this basic infrastructure by
illustrating the fact that this infrastructure can easily become disfunction by
natural disaster or corrupted government.
By giving the example of the inadequate investment on energy infrastructure from
British government in recent years, the author proves the point that no one
should take nowadays adequate supply infrastructure for granted. 
%
% Protect
%
%
%
% How do they contribute to social and cultural development?
% What significant example is provided? 
%
%




\newpage
\bibliographystyle{asmems4}
\bibliography{asme2e}


\end{document}
